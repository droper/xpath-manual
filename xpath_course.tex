%%%%%%%%%%%%%%%%%%%%%%%%%%%%%%%%%%%%%%%%%%%%%%%%%%%%%%%%%%%%%%%%%%%%%%%%%%%
%
% Clases de Xpath
%
%%%%%%%%%%%%%%%%%%%%%%%%%%%%%%%%%%%%%%%%%%%%%%%%%%%%%%%%%%%%%%%%%%%%%%%%%%%

\documentclass[12pt,a4paper,oneside, openany]{book}
\textheight=26cm
\textwidth=18cm
\topmargin=-2cm
\oddsidemargin=-1cm
\parindent=0mm

\usepackage[utf8x]{inputenc}
%\usepackage[spanish]{babel}
\usepackage{ucs}
\usepackage{amsmath,amsfonts,amssymb}
\usepackage{graphicx}
\usepackage{anyfontsize}
\usepackage{float}
\usepackage{textcomp}
%Paquete verbments para obtener bonito código fuente
\usepackage{verbments}
\usepackage{hyperref}
\usepackage{listings}

\fvset{frame=single,framerule=1pt}
%\plset{language=python ,texcl=true,style=vs,%
%captionfont=\sffamily\color{white}}

\setcounter{chapter}{0}

\pagestyle{plain}


%--------------------------------------------------------------------------
%titulo
\title{ 
 \begin{center}
  {Xpath}\\
 \end{center}
}

\author{Pedro Muñoz del Rio\\
        pmunoz@gmail.com}

\date{Lima, Perú}

\begin{document}

\renewcommand{\baselinestretch}{1.5} %espacio entre lineas, 2 es doble espacio
\renewcommand{\contentsname}{Index}
\renewcommand{\listfigurename}{List of Figures}
\renewcommand{\chaptername}{Chapter}
\renewcommand{\bibname}{Sources}
\renewcommand{\figurename}{Figure}
\renewcommand{\tablename}{Table}

 \maketitle %despliega el titulo
 
  \frontmatter  %Secciones del documento que van con numeros romanos en el indice 
 
   This course is designed to introduce you to Xpath, and how this technology is used in tasks like scraping and user interface testing. The goal is to provide the student with a deep understanding of Xpath and how to use it.\\

Since this course may be the only formal training many students will take in the topic, we will focus not only in the theory but apllying it with many exercises so the concepts are well understood.\\




 %Resumen y Abstract
  
 \tableofcontents %despliega el índice
 \listoffigures   %Tabla de figuras
      
 \mainmatter  %Cuerpo del documento  

 %Cuerpo del documento 

 \chapter{Xpath}
    \input what_is_xpath.tex

 \chapter{Working with Xpath}
    \input working_xpath.tex
   

   \bibliographystyle{style}
   \bibliography{bibliography}

\end{document}

